% Use only LaTeX2e, calling the article.cls class and 12-point type.
% !TeX spellcheck = en_AU
% Ser the document type and font
\documentclass[12pt]{report}

% graphics packages
\usepackage{graphicx}
\usepackage{pdflscape}
\graphicspath{ {images/} }

% tables packages
\usepackage{longtable}
\usepackage{makecell}


\usepackage[utf8]{inputenc}
\usepackage{color}
\usepackage{hyperref}
\usepackage{booktabs}
\setlength\parindent{0pt}
\usepackage{fourier} 
\usepackage{array}

%------------------------------------------------------------------
% Change the url link to blue color
%------------------------------------------------------------------
\DeclareUrlCommand{\url}{%
	\def\UrlFont{\color{blue}\normalfont}%      Adding a little color 
%	\def\UrlLeft##1\UrlRight{\underline{##1}}%  Underlining the url
}

% The following parameters seem to provide a reasonable page setup.
\topmargin 0.0cm
\oddsidemargin 0.2cm
\textwidth 16cm 
\textheight 21cm
\footskip 1.0cm

% start the document 
\begin{document} \sloppy

\author{Author Name}
\date{Day Month Year}

%------------------------------------------------------------------
% Load prerequisite packages 
%------------------------------------------------------------------


% CARP documentation
\begin{center}
	\section{CARP manual 1.1}
	\noindent\textbf{b\'iogo written by: Dan Kortschak} \\
	\textbf{Document organised by: Lu Zeng} \\
	\today
\end{center}

\noindent krishna and igor are \textit{ab initio} repeat family identifiction and annotation programs, that  identify repeat element boundaries and family relationships from whole-genome sequence data. These programs build, refine and classify consensus models of putative interspersed repeats. krishna and igor are built using b\'iogo \href{<url>}(\url{https://github.com/biogo/biogo/}), a bioinformatics library for the Go language. 

%------------------------------------------------------------------
% Load prerequisite packages 
%------------------------------------------------------------------

\subsection*{{Prerequisites}}

\subsubsection{ Download Go }
Available at \href{<url>} (\url{https://golang.org/dl}).\\
For installation details, follow the instructions on the \href{https://golang.org/doc/install#install} {\color{blue}Go installation} page. \\
{\color{red} Note:}  For convenience, add the workspace's bin subdirectory to your PATH, or add in .profile. 

\begin{enumerate}
	\item[*] export PATH=\$PATH:\$(go env GOPATH)/bin
\end{enumerate}

\subsubsection{Git}
To perform the next step you will need Git to be installed. (Check that you have a git command before proceeding.) \\

\noindent If you do not have a working Git installation, follow the instructions on the \href{https://git-scm.com/downloads}{\color{blue}Git download} page.

\subsubsection{Download b\'iogo Packages}
Download and install krishna and igor packages from github. \\

\begin{enumerate}
	\item[*] go get -u github.com/biogo/examples/krishna
	\item[*] go get github.com/biogo/examples/krishna/matrix
	\item[*] go get github.com/biogo/examples/igor
	\item[*] go get github.com/biogo/examples/igor/seqer
	\item[*] go get github.com/biogo/examples/igor/gffer
\end{enumerate}

\subsubsection{ Install CENSOR}
Install censor to screen target genomes against a reference collection of repeats with masking symbols, as well as generating a report classifying all repeats found. CENSOR needs wu-blast/NCBI-blast and Bioperl installed.\\

\noindent CENSOR is available at \href{http://www.girinst.org/downloads/software/censor/} {\color{blue}CENSOR download} page.

\subsubsection{ Install MUSCLE}
Install MUSCLE to generate consensus sequences.

\noindent MUSCLE is available at \href{https://www.drive5.com/muscle/downloads.htm} {\color{blue}MUSCLE download} page.
\noindent The installation information can be seen at \href{https://www.drive5.com/muscle/manual/install.html} {\color{blue}MUSCLE Install} page.

%------------------------------------------------------------------
% Run first step, krishna alignment and family classification
%------------------------------------------------------------------

\subsection*{Example run}
In this example, the human genome was downloaded as chromosomes (24 chromosomes) from UCSC into files called chr*.fa.\\

\subsubsection{Use krishna to do pairwise alignment between human genome sequences}

krishna-matrix helps you to align sequences by using a matrix table. \\

\noindent The default minimum hit length for krishna (-dplen) is 400bp, and minimum hit identity (-dpid) is 94\%. The smaller the length and the lower the hit identity parameters you use, the more time and memory you will need. \\

\noindent If you want to change running parameters, for example, minimum hit length with 200bp, and minimum hit identity with 90\%, just specify the parameters when running matrix {\color{red}-krishnaflags=\textquotedbl-tmp=/scratch -threads=8 -log -filtid=0.9 -filtlen=200\textquotedbl}.\\

\noindent Now run the job:

\begin{enumerate}
	\item[*] cd /data/rc003/lu/human (directory contains the sequences)
	\item[*] matrix -krishnaflags=\textquotedbl-tmp=/scratch -threads=8 -log -filtid=0.94 -filtlen=400\textquotedbl{} chr*.fa
\end{enumerate}

\noindent If genome sequence files are very big and consist of multiple contigs or scaffolds ($>$200MB), split them into smaller files (using bundle.go), please find all the relavant codes under \href{<url>}(\\url{https://github.com/luzengAdelaide/Biogo-document/tree/master/Codes}).\\

\begin{enumerate}
	\item[*] go build -o \$(go env GOPATH)/bin/bundle bundle.go
	\item[*] bundle -bundle 80000000 -in seq.fa
\end{enumerate}

\noindent -bundle: Specifies the total sequence length in a bundle. (default 20000000, 20MB).\\
-in: the genomes sequences you need to split.\\
Then run krishna job.\\

\subsubsection{Use igor to report repeat feature family groupings in JSON format.}

After running krishna, igor will take the pairwise alignment data to cluster repeat families.
\begin{enumerate}
	\item[*] cat *.gff \texttt{>} hg\_krishna.gff
	\item[*] igor -in hg\_krishna.gff -out hg94\_krishna.json
\end{enumerate}

\subsubsection{Use seqer to generate consensus sequences from genome intervals .}
seqer returns multiple fasta sequences corresponding to feature intervals described in the JSON output from igor. \\

gffer converts the JSON output of igor to gff. seqer will produce fastq consensus sequence output from either MUSCLE or MAFFT. 

\begin{enumerate}
	\item[*] gffer \texttt{<} hg94\_krishna.json \texttt{>} hg94\_krishna.igor.gff
	\item[*] cat chr*.fa \texttt{>} hg19v37.mfa
	\item[*] seqer -aligner=muscle -dir=consensus -fasta=true -maxFam=100 -subsample=true -minLen=0.95 -threads=12 -ref=hg19v37.mfa hg94\_krishna.igor.gff
\end{enumerate}

\noindent -fasta: Output consensus as fasta with quality case filtering\\
-maxFam: maxFam indicates maximum family size permitted (0 == no limit).\\
-minLen: Minimum proportion of longest family member.\\
-threads: Number of concurrent aligner instances to run.


\subsection{Benchmarks}

\footnotesize  % Switch from 12pt to 11pt; otherwise, table won't fit
%\setlength\LTleft{-50pt}            % default: \fill
%\setlength\LTright{-50pt}           % default: \fill
\setlength\tabcolsep{1.5pt}
\begin{center}
	\begin{tabular}{|l|c|c|c|c|c|c|}
		\hline
		\thead{Genome}	&	\thead{ Krishna Threads }	&	\thead{Genome DB \\ Size}	& \thead{Krishna run time \\ (hh:mm)}	&	\thead{Igor run time \\ (hh:mm)}	&	\thead{Seqer run time \\ (hh:mm)}  \\
		\hline
		Human	&	8	&	3.0G	&	~$\sim$200	&	128:30	&	2:23 \\
		\hline 
		Bearded Dragon	&	8	&	1.8G	&	~$\sim$23	&	73:11	&	$<$4 \\
		\hline
		Anolis	& 6	& 1.8G	& 76:52	&	97:32	& 2:40	\\
		\hline
		Chicken	&	4	&	1017M	&	5	&	$<$4 & $<$1 \\
		\hline
		Opossum	&	8	&	3.5G	&	~$\sim$83	&	61:48	&	4:52 \\
		\hline
		Platypus	&	8	&	2.0G	&	~$\sim$99	&	191:34	&	10:16 \\
		\hline
		Echidna	&	8	&	1.9G	&	$<$ 360	&	12:43	&	9:02 \\
		\hline
	\end{tabular}
\end{center}

* Analysis runs on a machine with 512GB RAM, running Red Hat Linux \\

%------------------------------------------------------------------
% Second step, repeat family annotation
%------------------------------------------------------------------

\subsection{Repeat Library Annotation}
Previous steps have generated repeat consensus sequences from the human genome, now we are going to annotate these repeat consensus sequences. \\
\textbf{\textcolor{red}{All the files used below can be found at \href{<url>}(\url{https://data.mendeley.com/datasets/k88h5xnhcb/edit})}}. \\
\textbf{\textcolor{red}{All the codes used below can be found at \href{<url>}(\url{https://github.com/luzengAdelaide/Biogo-document/tree/master/Codes})}.} \\



\subsection{Annotate consensus sequences}
\textbf{\textcolor{red}{Notes: All Java scripts used here need you to specify the directories where your input data is and where you want your output written. }}

\subsubsection{Annotate consensus sequences with repeat families.}
Use censor to annotate consensus sequences with the Repbase library. The Vertebrates.fa we use here is the Repbase vertebrates repeat libraries downloaded on 1st March, 2016. 
\begin{enumerate}
	\item[*] cd consensus
	\item[*] cat *.fq \texttt{>} ConsensusSequences.fa 
	\item[*] censor -bprm cpus=8 -lib $\sim$/Vertebrates.fa -lib ~$\sim$/our\_known\_reps\_20130520.fasta ConsensusSequences.fa 
\end{enumerate}
The censor output usually contains 5 files: ConsensusSequences.fa.map, ConsensusSequences.fa.aln, ConsensusSequences.fa.found, ConsensusSequences.fa.idx, ConsensusSequences.fa.masked.  \\

\subsubsection{Classify consensus sequences.}
ConsensusSequences.fa and ConsensusSequences.fa.map are required in this step. You will also need to specify the directories for your input data and where you want your output written in the java code. Edit the source, compile and run.\\
\begin{enumerate}
	\item[*] javac ClassifyConsensusSequences.java
	\item[*] java ClassifyConsensusSequences
\end{enumerate}
This should generate 5 output files: known.txt, partial.txt, check.txt, notKnown.fa, notKnown.fa.gff. \\
Then we need to further annotate these notKnown.fa consensus sequences. 

\subsubsection{Filter potential protein sequences.}
In this step, you will need to run the perl script reportsJ.pl (perl reportsJ.pl). This uses wu-blast/NCBI-blast blastx, you will also need the uniprot database. reportsJ.pl will ask for a list of databases to process, "libs.txt". This will include notknown.fa and notknown.fa.gff, put libs.txt into the current directory. \\

\noindent\textbf{\textcolor{red}{Notes: reportsJ.pl contains three parts: 1) identify uniprot, 2) GB\_TE, 3) retroviruses. You can comment out two parts, and run each part separately in parallel to save time. \\
		From these three steps, you will get three output files for following steps: 1) notKnown.fa.spwb.gff, 2) notKnown.fa.tewb.gff, 3) notKnown.fa.ervwb.gff}} 

\subsubsection{Get protein information from consensus sequences.}
Another java program GetProteins.java will be used. It needs two input files: notKnown.fa, notKnown.fa.spwb.gff (Generated from previous step).
\begin{enumerate}
	\item[*] javac GetProteins.java
	\item[*] java GetProteins
\end{enumerate}
You will get 2 output files: proteins.txt (a list of families that have been identified as proteins and the proteins they match);
notKnownNotProtein.fa (a fasta file of the families that were not classified).

\subsubsection{Check for simple sequence repeats (SSR).}
Check for existence of SSR in the unknown sequences, using phobos. Phobos can be downloaded at \href{<url>}(\url{http://www.ruhr-uni-bochum.de/ecoevo/cm/cm_phobos.htm}). We used executable: phobos-linux-gcc4.1.2\\
.
\begin{enumerate}
	\item[*] phobos-linux-gcc4.1.2 -r 7 --outputFormat 0 --printRepeatSeqMode 0 notKnownNotProtein.fa $>$ notKnownNotProtein.phobos
\end{enumerate}

\subsubsection{Identify the sequences that are SSRs from the phobos output.}
phobos output will be used to identify SSRs: notKnownNotProtein.phobos.
\begin{enumerate}
	\item[*] javac IdentifySSRs.java
	\item[*] java IdentifySSRs
\end{enumerate}
Your output will be a file called: SSR.txt

\subsubsection{Generate annotated repeat library.}
There are ten input files that are required to generate a repeat library: \\
1. ConsensusSequences.fa \\
2. ConsensusSequences.fa.map \\
3. notKnown.fa.tewb.gff \\
4. notKnown.fa.ervwb.gff \\
5. protein.txt \\
6. known.txt \\
7. GB\_TE.21032016.fa \\
8. all\_retrovirus.fasta \\
9. SSR.txt (if you do not have this, leave the definition in, it will generate error messages, but will not stop the program or affect the results.) \\
10. LA4v2-satellite.fa (you do not have this, or equivalent, as you didn't have any satellites, but leave the definition in-it will cause error messages, but will not stop the program or affect the results.)
\begin{enumerate}
	\item[*] javac GenerateAnnotatedLibrary.java
	\item[*] java GenerateAnnotatedLibrary
\end{enumerate}
This will generate a library called \textquotedblleft Human\_Repeat\_Library.fasta\textquotedblright, you can change this to whatever you want to call your library.

\subsection{Benchmarks2}

\footnotesize  % Switch from 12pt to 11pt; otherwise, table won't fit
%\setlength\LTleft{-50pt}            % default: \fill
%\setlength\LTright{-50pt}           % default: \fill
\setlength\tabcolsep{1.5pt}
\begin{center}
	\begin{tabular}{ | c | c | c | c| c |c|}
		\hline
		\thead{Genome}	& 	\thead{Consensus sequences size}	&	\thead{Censor first run \\ time (hh:mm)}	&	\thead{reportJ.pl \\ (hh:mm)}	&	\thead{phobos run time \\ (hh:mm)} \\
		\hline
		Human	&	38M	&	5:14	&	19:30	&	$<$00:10 \\
		\hline 
		Bearded Dragon	&	88M	&	22:21	&	$<$178	&	$<$00:10 \\
		\hline
		\textcolor{red}{New Bearded Dragon}	&	\textcolor{red}{88M}	&	\textcolor{red}{7:13}	&	\textcolor{red}{18:28}	&	\textcolor{red}{$<$00:10} \\
		\hline
		Anolis	&	63M	&	9:42	&	78	&	$<$00:10 \\
		\hline
		Chicken	&	18M	&	3:01	&	$<$24	&	$<$00:10 \\
		\hline
		Opossum	&	60M	&	13:17	&	80	&	$<$01:00 \\
		\hline
		Platypus	&	162M	&	17:34	&	115	&	$<$01:00 \\
		\hline
		Echidna	&	59M	&	18:40	&	105	&	01:54 \\
		\hline
	\end{tabular}
\end{center}

\noindent * Analysis run on a slurm machine with 4$\sim$16 cpus, and 8GB RAM, running Red Hat Linux.\\
\textcolor{red}{** New Bearded dragon analysis used same bearded dragon genome, except it was run on a High Performance Computing machine with 32 cpus, running Red Hat Linux.} \\


\textbf{\textcolor{blue}{ \Large Finished! Good Luck!!!}}

\end{document}